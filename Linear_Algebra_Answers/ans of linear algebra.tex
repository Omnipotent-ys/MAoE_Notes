\documentclass{ctexart}
\usepackage{amsmath, amsthm, amssymb, graphicx, geometry, arydshln, mdframed}
\geometry{left=2.54cm , right=2.54cm , top=3.17cm , bottom=3.17cm}

\title{Liner Algebra Answers}
\author{Omniporent-ys}
\date{\today}

\begin{document}
    \maketitle

    \begin{abstract}
        63012409班一名同学对北京理工大学《线性代数B》的答案编写
    \end{abstract}
    \section{习题一答案}
    \section{习题二答案}
    \section{习题三答案}
        \subsection{}
            \begin{mdframed}
            3$\vec{\alpha}$ + 4$\vec{\beta}$
            = 3$\begin{bmatrix}
                1 \\
                2 \\
                3 \\
                4\\           
               \end{bmatrix}$
               + 4$\begin{bmatrix}
                0 \\
                2 \\
                4 \\
                6 \\           
               \end{bmatrix}$
             = $\begin{bmatrix}
                3 \\
                6 \\
                9 \\
                12 \\           
               \end{bmatrix}$
               +
               $\begin{bmatrix}
                0 \\
                8 \\
                16 \\
                24 \\           
               \end{bmatrix}$
                = 
                $\begin{bmatrix}
                3 \\
                14 \\
                25 \\
                36 \\           
               \end{bmatrix}$
            \end{mdframed}
        
        \subsection{}
            \begin{mdframed}
                2$\vec{\alpha}_1$ + 2$\vec{\beta}$ + 3$\vec{\alpha}_2$ - 3$\vec{\beta}$
                 = 2$\vec{\alpha}_3$ + 2$\vec{\beta}$
                 \\$\vec{\beta}$ = $\frac{2}{3} \vec{\alpha}_1$ + $\vec{\alpha}_2$ - $\frac{2}{3} \vec{\alpha}_3$
                 = $\begin{bmatrix}
                    ~\frac{5}{3}\\
                    ~\frac{7}{3}\\
                    ~\frac{5}{3}\\
                    -\frac{4}{3}\\
                    \end{bmatrix}$            
            \end{mdframed}

        \subsection{}
            \begin{mdframed}
                取$\vec{\alpha}$ = 
                $\begin{bmatrix}
                    x_1,x_2,\cdots,x_n\\
                \end{bmatrix}^{T}$ $\in$ $\mathbf{V}$
                \\ $\vec{\beta}$ = 
                $\begin{bmatrix}
                    y_1,y_2,\cdots,y_n\\
                \end{bmatrix}^T$ $\in$ $\mathbf{V}$
                则有$\vec{\alpha}+\vec{\beta}$ = 
                $\begin{bmatrix}
                    x_1+y_1 , x_2+y_2 , \cdots , x_n+y_n\\
                \end{bmatrix}^{T}$ $\in$ $\mathbf{V}$\\
            设$x_i - a_i = c_x , y_i - a_i = c_y$,
              $x_i + y_i - a = c_{xy}$\\
            则有$a_i = c_{xy}-c_x-c_y$,即$a_i$的值与i无关\\
            这说明$a_1=a_2=\cdots=a_n$\\
            $k\vec{\alpha}\in\mathbf{V}$请自行验证
            \end{mdframed}
            
        \subsection{}
            \begin{mdframed}
                (1)取$\vec{\alpha}$=$\begin{matrix}
                    1,1
                \end{matrix}^{T}$
                $\in\mathbf{V_1}$
            \end{mdframed}







            设$(\mathbf{A}-2\mathbf{I})(\mathbf{A}+n_1\mathbf{I})=n_2\mathbf{I}$,
            
            展开并整理可以得到:
            $\mathbf{A}^2+(n_1-2)\mathbf{A}+(-2n_1-n_2)\mathbf{I}=\mathbf{0}$,
            其中$\begin{cases}
                n_1-2=-2\\
                -2n_1-n_2=-1
            \end{cases}$
            
            解得:$(\mathbf{A}-2\mathbf{I})\mathbf{A}=\mathbf{I}$,
            
            所以$(\mathbf{A}-2\mathbf{I})^{-1} = \mathbf{A}$,证毕。
            
        对于一个矩阵A,如果A的秩为r,则有:
        
        \textbf{\fangsong r(A) = 行阶梯型矩阵的非零行行数 = 有效方程的个数 = 独立向量的个数 = 非零子式的最高阶数}
        
        从行的角度来看,矩阵是线性方程组;而从列的角度来看,矩阵是列向量的集合。如此,
        从不同的角度我们都可以理解rank这个概念。
        \subsection{分块矩阵}
            \subsubsection{对角线的可逆矩阵}
            设$\mathbf{A}$是$m$阶的方阵,$\mathbf{D}$是$n$阶的方阵。如果$A$和$D$都是可逆的,那么
            $\begin{bmatrix}
                \mathbf{A} & 0 \\
                0 & \mathbf{D} \\
            \end{bmatrix}$是可逆矩阵。进一步地,
            $
            \begin{bmatrix}
                \mathbf{A} & 0 \\
                0 & \mathbf{D} \\
            \end{bmatrix}^{-1} = \begin{bmatrix}
                \mathbf{A}^{-1} & 0 \\
                0 & \mathbf{D}^{-1} \\
            \end{bmatrix}
            $。然而,这个结论不适用于反对角线矩阵。
            \subsubsection{反对角线矩阵的可逆矩阵}
            设$\mathbf{A}$是$m$阶的方阵,$\mathbf{D}$是$n$阶的方阵。如果$A$和$\mathbf{D}$都是可逆的,那么
            $\begin{bmatrix}
                0 & \mathbf{A} \\
                \mathbf{D} & 0 \\
            \end{bmatrix}$是可逆矩阵。进一步地,
            $
            \begin{bmatrix}
                0 & \mathbf{A} \\
                \mathbf{D} & 0 \\
            \end{bmatrix}^{-1} = \begin{bmatrix}
                0 & \mathbf{D}^{-1} \\
                \mathbf{A}^{-1} & 0 \\
            \end{bmatrix}
            $。可以观察到,这里矩阵$\mathbf{A}$与$\mathbf{D}$在取逆矩阵的时候进行了“对调”,这是一个很有意思的现象,在实际应用中也需要格外注意。
    \section{典型例题}
            \subsection{矩阵}
            \begin{itemize}
                \item \textbf{例题1}:设$\mathbf{A}$是一个$n$阶的方阵,且$\mathbf{A}^2~-2\mathbf{A}~-\mathbf{I}~=~\mathbf{0}$,求$(\mathbf{A}-2\mathbf{I})^{-1}$
            \end{itemize}
            
            {\fangsong 类似的提醒一般都需要因式分解,且常用待定系数法来解决问题}\\
            \begin{mdframed}
            解:

            设$(\mathbf{A}-2\mathbf{I})(\mathbf{A}+n_1\mathbf{I})=n_2\mathbf{I}$,
            
            展开并整理可以得到:
            $\mathbf{A}^2+(n_1-2)\mathbf{A}+(-2n_1-n_2)\mathbf{I}=\mathbf{0}$,
            其中$\begin{cases}
                n_1-2=-2\\
                -2n_1-n_2=-1
            \end{cases}$
            
            解得:$(\mathbf{A}-2\mathbf{I})\mathbf{A}=\mathbf{I}$,
            
            所以$(\mathbf{A}-2\mathbf{I})^{-1} = \mathbf{A}$,证毕。
            
            \end{mdframed}

            在这个题中,用待定系数法显得比较繁琐,可以直接通过移项看出结果,但是这里介绍的是通法,是一种思想,其理论基础是:
            $$ \mathbf{AA^{-1}}~=~\mathbf{I}$$
            \vrule width11cm height1.5pt depth-1pt %这是一条线
            %需要注意的是,这里要使用虚线和实线,必须引入arydshln包,而且必须在array之后引入
            
            https://zhuanlan.zhihu.com/p/266267223
            $$
            \begin{array}{c|c:c}
                1 & 4 & 5\\
                1 & 6 & 10\\
                2 & 5 & 11\\
            \end{array}
            $$

            %\texttt{https://blog.csdn.net/qq\_46673125/article/details/141362565}

            https://www.zhihu.com/question/362654946/answer/2364047739
            
\end{document}\documentclass{ctexart}
\usepackage{amsmath, amsthm, amssymb, graphicx, geometry, arydshln, mdframed}
\geometry{left=2.54cm , right=2.54cm , top=3.17cm , bottom=3.17cm}

\title{Liner Algebra Notes}
\author{lamaper}
\date{\today}

\begin{document}
    \maketitle

    \begin{abstract}
        这是线性代数B这门课的学习笔记,主要记录一些关键的习题和知识点,以及做题时的一些思路。同时
        也是对Latex的一次练习。
    \end{abstract}

    \section{常用定理与推论}
        \subsection{矩阵的秩rank}
        对于一个矩阵A,如果A的秩为r,则有:
        
        \textbf{\fangsong r(A) = 行阶梯型矩阵的非零行行数 = 有效方程的个数 = 独立向量的个数 = 非零子式的最高阶数}
        
        从行的角度来看,矩阵是线性方程组;而从列的角度来看,矩阵是列向量的集合。如此,
        从不同的角度我们都可以理解rank这个概念。
        \subsection{分块矩阵}
            \subsubsection{对角线的可逆矩阵}
            设$\mathbf{A}$是$m$阶的方阵,$\mathbf{D}$是$n$阶的方阵。如果$A$和$D$都是可逆的,那么
            $\begin{bmatrix}
                \mathbf{A} & 0 \\
                0 & \mathbf{D} \\
            \end{bmatrix}$是可逆矩阵。进一步地,
            $
            \begin{bmatrix}
                \mathbf{A} & 0 \\
                0 & \mathbf{D} \\
            \end{bmatrix}^{-1} = \begin{bmatrix}
                \mathbf{A}^{-1} & 0 \\
                0 & \mathbf{D}^{-1} \\
            \end{bmatrix}
            $。然而,这个结论不适用于反对角线矩阵。
            \subsubsection{反对角线矩阵的可逆矩阵}
            设$\mathbf{A}$是$m$阶的方阵,$\mathbf{D}$是$n$阶的方阵。如果$A$和$\mathbf{D}$都是可逆的,那么
            $\begin{bmatrix}
                0 & \mathbf{A} \\
                \mathbf{D} & 0 \\
            \end{bmatrix}$是可逆矩阵。进一步地,
            $
            \begin{bmatrix}
                0 & \mathbf{A} \\
                \mathbf{D} & 0 \\
            \end{bmatrix}^{-1} = \begin{bmatrix}
                0 & \mathbf{D}^{-1} \\
                \mathbf{A}^{-1} & 0 \\
            \end{bmatrix}
            $。可以观察到,这里矩阵$\mathbf{A}$与$\mathbf{D}$在取逆矩阵的时候进行了“对调”,这是一个很有意思的现象,在实际应用中也需要格外注意。
    \section{典型例题}
            \subsection{矩阵}
            \begin{itemize}
                \item \textbf{例题1}:设$\mathbf{A}$是一个$n$阶的方阵,且$\mathbf{A}^2~-2\mathbf{A}~-\mathbf{I}~=~\mathbf{0}$,求$(\mathbf{A}-2\mathbf{I})^{-1}$
            \end{itemize}
            
            {\fangsong 类似的提醒一般都需要因式分解,且常用待定系数法来解决问题}\\
            \begin{mdframed}
            解:

            设$(\mathbf{A}-2\mathbf{I})(\mathbf{A}+n_1\mathbf{I})=n_2\mathbf{I}$,
            
            展开并整理可以得到:
            $\mathbf{A}^2+(n_1-2)\mathbf{A}+(-2n_1-n_2)\mathbf{I}=\mathbf{0}$,
            其中$\begin{cases}
                n_1-2=-2\\
                -2n_1-n_2=-1
            \end{cases}$
            
            解得:$(\mathbf{A}-2\mathbf{I})\mathbf{A}=\mathbf{I}$,
            
            所以$(\mathbf{A}-2\mathbf{I})^{-1} = \mathbf{A}$,证毕。
            
            \end{mdframed}

            在这个题中,用待定系数法显得比较繁琐,可以直接通过移项看出结果,但是这里介绍的是通法,是一种思想,其理论基础是:
            $$ \mathbf{AA^{-1}}~=~\mathbf{I}$$
            \vrule width11cm height1.5pt depth-1pt %这是一条线
            %需要注意的是,这里要使用虚线和实线,必须引入arydshln包,而且必须在array之后引入
            
            https://zhuanlan.zhihu.com/p/266267223
            $$
            \begin{array}{c|c:c}
                1 & 4 & 5\\
                1 & 6 & 10\\
                2 & 5 & 11\\
            \end{array}
            $$

            %\texttt{https://blog.csdn.net/qq\_46673125/article/details/141362565}

            https://www.zhihu.com/question/362654946/answer/2364047739
            
\end{document}