\documentclass[12pt , a4paper , oneside]{ctexart}
\usepackage{amsmath, amsthm, amssymb, graphicx, geometry, mdframed, booktabs}
\geometry{left=2.54cm , right=2.54cm , top=3.17cm , bottom=3.17cm}

%introduction

\title{工科数学分析(上)学习笔记}
\author{lamaper}
\date{\today}

\begin{document}
    \maketitle

    
    \section{前置知识}
        \subsection{常用数列求和}
        $$1^2 + 2^2 + ...+n^2 = \frac{1}{6}n(n+1)(2n+1)$$
        $$1^3 + 2^3 + ...+n^3 = \frac{1}{4}n^2(n+1)^2$$  

        \subsection{和差化积公式}
        $$\sin \alpha + \sin \beta = 2\sin (\frac{\alpha + \beta}{2})\cos (\frac{\alpha - \beta}{2})$$
        $$\sin \alpha - \sin \beta = 2\cos (\frac{\alpha + \beta}{2})\sin (\frac{\alpha - \beta}{2})$$
        $$\cos \alpha + \cos \beta = 2\cos (\frac{\alpha + \beta}{2})\cos (\frac{\alpha - \beta}{2})$$
        $$\cos \alpha - \cos \beta = -2\sin (\frac{\alpha + \beta}{2})\sin (\frac{\alpha - \beta}{2})$$

        \subsection{积化和差公式}
        $$\sin \alpha \cos \beta = \frac{1}{2}[\\sin (\alpha + \beta) + \\sin (\alpha - \beta)]$$
        $$\cos \alpha \sin \beta = \frac{1}{2}[\sin (\alpha + \beta) - \sin (\alpha - \beta)]$$
        $$\cos \alpha \cos \beta = \frac{1}{2}[\cos (\alpha + \beta) + \cos (\alpha - \beta)]$$
        $$\sin \alpha \sin \beta = -\frac{1}{2}[\cos (\alpha + \beta) - \cos (\alpha - \beta)]$$

    \section{函数、极限与连续}
        \subsection{常用定理与推论}
            \subsubsection{一些简单的极限}
                \begin{itemize}
                \item $\lim\limits_{n \to +\infty} \frac{1}{n} = 0$
                \item $\lim\limits_{n \to +\infty} q^n = 0 (|q| < 1)$
                \item $\lim\limits_{n \to +\infty} n^\frac{1}{n} = 1$
                \end{itemize}

            \subsubsection{归并性}
            例如要证明$\lim\limits_{x \to + \infty} \sin x$不存在,可以使用归并性证明。

            证明如下:

            令$f(x)=\sin x$,如果取$x_n=n\pi$($x_n$单调增加趋于正无穷大),则有

            $$\lim\limits_{x \to + \infty} f(x_n) = \lim\limits_{x \to + \infty} \sin n\pi = 0$$

            然后又取$x_n=\frac{\pi}{2}+2n\pi$($x_n$单调增加趋于正无穷大),则有

            $$\lim\limits_{x \to + \infty} f(x_n) = \lim\limits_{x \to + \infty} \sin (\frac{\pi}{2}+2n\pi) = 1$$
            
            所以$\lim\limits_{x \to + \infty} \sin x$不存在。
            
            \subsubsection{一种证明极限的思路}
            要证明$f(x)$极限,则要证明$\lim\limits_{x \to \infty} f(x) =A$

            那么设$f(x)=A+a(x)$,那么就要证明$\lim\limits_{x \to \infty} a(x) = 0$

            这是一种利用无穷小性质来证明极限的方法。

            \subsubsection{无穷小定理}
            \begin{itemize}
                \item 有界函数与无穷小的乘积是无穷小
                \item 有限个无穷小的和是无穷小
                \item 有限个无穷小的乘积是无穷小
            \end{itemize}
            需要注意的的是,无穷小是变量,而不是一个很小的数。

            显然的,若$f(x)$是无穷小,$\frac{1}{f(x)}$是无穷大,反之亦然。

            \subsubsection{夹逼定理}
            如果$\lim\limits_{x \to x_0} f(x) = A$,$\lim\limits_{x \to x_0} g(x) = A$,且$f(x) \leq h(x) \leq g(x)$,则$\lim\limits_{x \to x_0} h(x) = A$

            利用夹逼定理可以证明一些不好直接说明的极限,它常与无穷小定理一起使用,同时伴有数学归纳法。

            \subsubsection{两个重要极限和常用等价无穷小}

            两个重要极限实际上代表两种特殊的极限形式:
            \begin{itemize}
                \item $\lim\limits_{x \to 0} \frac{\\sin  x}{x} = 1$ ($\frac{0}{0}$ 类型)
                \item $\lim\limits_{x \to 0} (1+x)^\frac{1}{x} = e$ ($1^\infty$ 类型)
            \end{itemize}

            在求解极限时,所看到的式子进行分析,确定类型,然后朝向这两个重要极限进行转化。

            重要极限也有一些扩展,如$\lim\limits_{x \to 0} \frac{a^x-1}{x} = \ln a~~(a \neq 0)$.

            当$x \to 0$时,有以下等价无穷小:
            \begin{itemize}
                
            \item  $x \sim \sin  x \sim \tan x \sim \arctan x \sim \arcsin  x$;
            
            \item  $1-\cos  x \sim \frac{x^2}{2}$;

            \item  $\ln(1+x) \sim e^x - 1 \sim x$;

            \item  $(1+x)^a -1 \sim ax$(a是非零常数);

            \item  $\frac{a^x-1}{x} \sim \ln a$;
            \end{itemize}

            \subsubsection{求极限策略}
            \begin{center}
            \begin{tabular}{c|c|c}
                \toprule
                \textbf{类型} & \textbf{操作} & \textbf{最终形式} \\  % Table header
                \midrule
                $\frac{0}{0}$或$\frac{\infty}{\infty}$ & 直接计算 &  \\  %第一行数据
                $0 \cdot \infty$ & 恒等变化 & $\frac{0}{\frac{1}{\infty}}$或者$\frac{\infty}{\frac{1}{0}}$\\  %第二行数据
                $\infty - \infty$ & 通分 & $\frac{\infty}{\infty}$或者$\frac{0}{0}$ \\  %第三行数据
                $\infty ^ 0$或者$0^0$或者$1^\infty$ & 朗博同构:$\lim u^v = e^{\lim(v\ln u)}$ & $\frac{0}{\frac{1}{\infty}}$或者$\frac{\infty}{\frac{1}{0}}$\\
                \bottomrule
            \end{tabular}
            \end{center}

        \subsection{典型例题}
            \subsubsection{利用数学归纳法证明}
            \textbf{证明下列数列有极限且求出极限:}

            (1)$y_1=10,y_n+1 = \sqrt{6+y_n},(n=1,2,...)$;

            (2)$y_1=\sqrt{2},y_n+1 = \sqrt{2 y_n},(n=1,2,...)$;

            \begin{mdframed}
            \textbf{解:}

            (1)先证明$y_n$有界,再证明$y_n$单调,最后说明$y_n$有极限。

            由表达式知道$y_n > 0$,$y_1 = 10 , y_2 = \sqrt{6 + 10} = 4 ,y_3 = \sqrt{6 + 4} = \sqrt{10}$

            观察猜测$y_n$应当不断减小,且减小趋势越来越小,所以$y_n$应当有下界且单调递减。

            接下来先证明有界:

            $y_1 = 10 > 3 ,y_2 = 4 > 3 , y_3 = \sqrt{10} > 3$,从而猜测$y_n > 3$,

            由数学归纳法,假设$y_n > 3$,则$y_n+1 = \sqrt{6+y_n} > \sqrt{6+3} = 3$,

            从而证明了$y_n$有下界。

            再证明$y_n$单调:

            $y_{n+1} - y_n = \sqrt{6+y_n} - y_n = \frac{6+y_n - y_n^2}{\sqrt{6+y_n} + y_n} = \frac{(2+y_n)(3-y_n)}{\sqrt{6+y_n} + y_n}$

            由于$y_n > 3$,所以$y_{n+1} - y_n < 0$,从而证明了$y_n$单调递减。

            \textbf{(技巧:面对根式相减,可以利用平方差公式构造恒正的分母,用无根号的分子来判断整个式子的正负)}

            最后求极限:

            设$\lim\limits_{n \to \infty} y_n = A$,则有$\lim\limits_{n \to \infty} y_{n+1} = \lim\limits_{n \to \infty} \sqrt{6+y_n} = \sqrt{6+A}$

            即$A=\sqrt{A+6}$,通过解方程可得$A=3$或$A=+2$,

            但是由于$y_n > 3 > 0$,所以舍去负根

            所以$\lim\limits_{n \to \infty} y_n = 3$

            \end{mdframed}

            \begin{mdframed}
            \textbf{解:}

            (2)仿照(1)的方法,先证明$y_n$有界,再证明$y_n$单调,最后说明$y_n$有极限。

            $y_1=\sqrt{2} < 2,y_2=\sqrt{2\sqrt{2}} < 2$,猜测$y_n < 2$,

            由数学归纳法,则$y_{n+1} = \sqrt{2y_n} < \sqrt{2 \cdot 2} = 2$,

            从而证明了$y_n$有上界。

            再证明$y_n$单调:

            $y_{n+1} - y_n = \sqrt{2y_n} - y_n = \frac{2y_n - y_n^2}{\sqrt{2y_n} + y_n} = \frac{y_n(2-y_n)}{\sqrt{2y_n} + y_n}$

            由于$y_n < 2$,所以$y_{n+1} - y_n > 0$,从而证明了$y_n$单调递增。

            最后求极限:

            设$\lim\limits_{n \to \infty} y_n = A$,则有$\lim\limits_{n \to \infty} y_{n+1} = \lim\limits_{n \to \infty} \sqrt{2y_n} = \sqrt{2A}$

            即$A=\sqrt{2A}$,通过解方程可得$A=2$

            \end{mdframed}

            \subsubsection{已知极限值求参数}
            这类题通常需要到一个结论:

            \begin{center}
                \fbox{$\lim\limits_{x \to \infty} f(x)g(x) = 0$,若$\lim\limits_{x \to \infty} f(x) = +\infty$,则$\lim\limits_{x \to \infty} g(x) = 0$}
            \end{center}

            \textbf{求出下列式子中$\alpha$与$\beta$的值:}

            (1)若$\lim\limits_{x \to +\infty} (x\arctan x - \alpha x - \beta) = 0$;

            (2)若$\lim\limits_{x \to +\infty} (\sqrt[3]{1-x^{3}}-\alpha x - \beta) = 0$;
            
            \begin{mdframed}
            (1) \textbf{解:}

            若$\lim\limits_{x \to +\infty} (x\arctan x - \alpha x - \beta) = 0$,
            则有$\lim\limits_{x \to +\infty} x(\arctan x - \alpha x - \frac{\beta}{x}) = 0$.

            由于$\lim\limits_{x \to +\infty} x = +\infty$,所以$\lim\limits_{x \to +\infty} (\arctan x - \alpha x - \frac{\beta}{x}) = 0$.

            由于$\lim\limits_{x \to +\infty} \arctan x = \frac{\pi}{2}$,
            $\lim\limits_{x \to +\infty} \frac{\beta}{x} = 0$,
            所以$\alpha = \frac{\pi}{2}$.

            接下来回代$\alpha = \frac{\pi}{2}$,则有$\lim\limits_{x \to +\infty} (x\arctan x - \frac{\pi}{2} x)=\beta$.

            设$u=\arctan x,x = \tan u$,即求$\lim\limits_{u \to \frac{\pi}{2}} - \tan u(\frac{\pi}{2} - u)$.
            为了方便可以再次换元$x = \frac{\pi}{2} - u$,即求$\lim\limits_{x \to 0} - x \tan(\frac{\pi}{2} - x)$.
            即$\lim\limits_{x \to 0} - \frac{x}{\cot x} = \lim\limits_{x \to 0} - \frac{x}{\sin x}\cos x = -1$.

            所以$\beta = -1$.
            
            综上,$\alpha = \frac{\pi}{2},\beta = -1$.
            \end{mdframed}
            
            \begin{mdframed}
            (2) \textbf{解:}
            若$\lim\limits_{x \to +\infty} (\sqrt[3]{1-x^{3}}-\alpha x - \beta) = 0$,
            则有$\lim\limits_{x \to +\infty} x (\sqrt[3]{ \frac{1}{x^3} - 1}-\alpha- \frac{\beta}{x}) = 0$.

            即$\lim\limits_{x \to +\infty} (\sqrt[3]{ \frac{1}{x^3} - 1}-\alpha- \frac{\beta}{x}) = 0$,
            可以求得$\alpha = -1$. 
            
            回代$\alpha = -1$,则有$\lim\limits_{x \to +\infty} (\sqrt[3]{1-x^{3}}+x)=\beta$.

            \textbf{\fangsong (技巧:面对根式,可以利用完全立方公式构造分母,便于求极限)}

            $\beta = \lim\limits_{x \to +\infty} (\sqrt[3]{1-x^{3}}+x) =
            \lim\limits_{x \to +\infty} -x (\sqrt[3]{ 1 + \frac{-1}{x^3} }-1)$,这里可以用等价无穷小
            $(1+x)^a -1 \sim ax$替换,则原式变为$\lim\limits_{x \to +\infty} -x\frac{1}{x^3} = 
            \lim\limits_{x \to +\infty} -\frac{1}{x^2} = 0$
            
            综上,$\alpha = -1,\beta = 0$.
            \end{mdframed}


\end{document}